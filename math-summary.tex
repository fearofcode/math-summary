\documentclass{article}

\usepackage{amsthm}
\usepackage{amsmath}
\usepackage{amssymb}

\usepackage{amsthm}
\makeatletter
\def\th@plain{%
  \thm@notefont{}% same as heading font
  \itshape % body font
}
\def\th@definition{%
  \thm@notefont{}% same as heading font
  \normalfont % body font
}
\makeatother

\begin{document}

\title{Math review for machine learning}
\author{Warren Henning\\\texttt{warren.henning@gmail.com}}
\date{\today}

\maketitle

\section{Disclaimer}

I haven't taken a university math course in several years. There may very well
be typos and outright mathematical errors!

Strictly speaking, I have no business writing this.

\section{Introduction}

This document is intended to summarize the essential math needed to understand
machine learning at a deep level. It is intended to summarize key formulas and
definitions, not replace actual texts.

\section{Calculus}

The following is taken from Spivak's {\it Calculus} \cite{spivak}.

\subsection{Limits}

\theoremstyle{definition}
\newtheorem{definition}{Definition}
\newtheorem{example}{Example}

\begin{definition}[$(\epsilon, \delta)$ definition of limit]

Let $f: \mathbb{R} \to \mathbb{R}$ be a function. 

\begin{equation}
\lim_{x \to a} f(x) = L
\end{equation}

if for every $\epsilon > 0$ there exists $\delta > 0$ such that if $0 < |x-a| <
\delta$, then $|f(x) - L| < \epsilon$. $\blacksquare$
 
\end{definition}

This formalizes the idea of being able to take $f(x)$ as close to $L$ as we want
by making $x$ sufficiently close to $a$.

\subsection{Continuity}

\begin{definition}

$f$ is continuous at $a$ if

\begin{equation}
lim_{x \to a} f(x) = f(a)
\end{equation}.

\end{definition}

\subsection{Derivatives}

\subsection{Integration}

\section{Linear Algebra}

\section{Analysis}

\section{Probability}

\section{Statistics}

\section{License}

This document is freely shareable under the terms of the GNU Free Documentation
license; see LICENSE in this document's directory for more information.

\section{Books}

\begin{thebibliography}{9}

\bibitem{spivak}
  Michael Spivak,
  \emph{Calculus}.
  Publish or Perish,
  4th Edition,
  2008.

\end{thebibliography}

\end{document}

